%
% 公立はこだて未来大学卒業研究中間報告書[全コース対応版]
%
%         ファイル名:"sample.tex"
%
\documentclass[11pt]{jarticle}
\usepackage{funinfosys}
\usepackage{url}
\usepackage[dvipdfmx]{graphicx}
\author{% 
b10xxxxx 未来太郎\\指導教員 : 函館一郎
}
\course{Information Systems Course}
%\course{Advanced ICT Course} %% 高度ICTコースの場合はこちらを使用
%\course{Information Design Course} %% 情報デザインコースの場合はこちらを使用
%\course{Complex Systems Course} %% 複雑系科学コースの場合はこちらを使用
%\course{Intelligent Systems Course} %% 知能システムコースの場合はこちらを使用
\title{中間報告書の書き方あ}
\etitle{How to Write Manuscripts for Midterm Report}
\eauthor{Taro MIRAI}
\abstract{和文は300から400文字で記述すること.}
\keywords{北海道, 函館, 亀田中野, 公立はこだて未来大学}
\eabstract{English should be written in 100 to 150 words.}
\ekeywords{Hokkaido, Hakodate, Kamedanakano, FUN}
\begin{document}
\maketitle
%\vspace*{-.5cm}

% 背景と目的
\input{contents/background_purpose.tex}

% 研究の位置づけ
\input{contents/positioning_of_research.tex}

% 関連研究
\input{contents/related_work.tex}

% 提案する理論
\input{contents/propose_method.tex}

% 実験と評価
\input{contents/experiments.tex}

% 考察
\input{contents/discussion.tex}

% 結言
\input{contents/conclusion.tex}

% 参考文献
\input{contents/reference.tex}

\end{document}
%
%
% EOF 
